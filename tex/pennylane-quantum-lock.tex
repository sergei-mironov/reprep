\section{Setup}

We are walking through the
\u{Building a quantum lock using phase kickback}{https://pennylane.ai/qml/demos/tutorial_phase_kickback}
tutorial by Danial Motlagh.

\begin{python}
import pennylane as qml
import numpy as np
import matplotlib.pyplot as plt
qml.drawer.use_style("black_white")
\end{python}

\begin{result}
\end{result}

\section{Implementation}

\begin{python}
import pennylane as qml
import numpy as np

num_wires = 5
dev = qml.device("default.qubit", wires=num_wires, shots=1)

def quantum_lock(secret_key):
  return qml.FlipSign(secret_key, wires=list(range(1, num_wires)))

def build_key(key):
  return qml.BasisState(key, wires=list(range(1, num_wires)))

@qml.qnode(dev)
def quantum_locking_mechanism(lock, key):
  build_key(key)
  qml.Hadamard(wires=0)  # Hadamard on ancilla qubit
  qml.ctrl(lock, control=0)  # Controlled unitary operation
  qml.Hadamard(wires=0)  # Hadamard again on ancilla qubit
  return qml.sample(wires=0)

def check_key(lock, key):
  if quantum_locking_mechanism(lock, key) == 1:
    print("Great job, you have uncovered the mysteries of the quantum universe!")
  else:
    print("Nice try, but that's not the right key!")

qml.draw_mpl(quantum_locking_mechanism)(lock, secret_key)
plt.savefig('img/fig.pdf')
\end{python}

\begin{result}

\end{result}

\begin{center}
\includegraphics[width=0.5\linewidth]{img/fig.pdf}
\end{center}

Running the experiment

\begin{python}
secret_key = np.array([0, 1, 1, 1])
lock = quantum_lock(secret_key)
check_key(lock, secret_key)
\end{python}

\begin{result}
Great job, you have uncovered the mysteries of the quantum universe!
\end{result}


\section{Discussion}

\begin{ai}
Hi Ai. Lets discuss this quantum program in PennyLane. It illustrates the Phase kickback.

\begin{python}
num_wires = 5
dev = qml.device("default.qubit", wires=num_wires, shots=1)

def quantum_lock(secret_key):
  return qml.FlipSign(secret_key, wires=list(range(1, num_wires)))

def build_key(key):
  return qml.BasisState(key, wires=list(range(1, num_wires)))

@qml.qnode(dev)
def quantum_locking_mechanism(lock, key):
  build_key(key)
  qml.Hadamard(wires=0)  # Hadamard on ancilla qubit
  qml.ctrl(lock, control=0)  # Controlled unitary operation
  qml.Hadamard(wires=0)  # Hadamard again on ancilla qubit
  return qml.sample(wires=0)
\end{python}

Can you try to generate an equation corresponding to this circuit?
\end{ai}

\begin{airesult}
Sure, let's analyze the quantum program you've presented and express its operations as an equation.
This code is implementing a variant of the phase kickback mechanism using controlled operations.
We'll break down the steps:

\begin{enumerate}
\item **Initialization**: It initializes the key via the \verb|BasisState| operation.
\item **Hadamard Gate**: It applies a Hadamard gate to the first qubit, putting it into
superposition.
\item **Controlled Operation**: It applies a controlled operation (\verb|ctrl|) which depends on the
secret lock function.
\item **Final Hadamard Gate**: It applies another Hadamard gate to the first qubit before sampling
it.
\end{enumerate}

The circuit can be described by the following sequence of operations:

\[
\text{Hadamard} \rightarrow \text{Controlled-Lock} \rightarrow \text{Hadamard}
\]

Given that the \textit{quantum lock} is represented by a \verb|FlipSign| operation, we can express
the effect on the initial state of the system. Using \(|\psi\rangle\) to denote the initial state of
the ancillary qubits representing the key, and assuming the control qubit is initially in the state
\(|0\rangle\), the transformation is given by:

\vsp

\[
\text{Initial State: } |0\rangle \otimes |\psi\rangle
\]

Applying the Hadamard on the control qubit:

\[
H|0\rangle = \dfrac{1}{\sqrt{2}}(|0\rangle + |1\rangle)
\]

Controlled operation with \verb|FlipSign|, dependent on whether the control qubit is \(|1\rangle\):

\[
\dfrac{1}{\sqrt{2}}(|0\rangle \otimes |\psi\rangle + |1\rangle \otimes
\text{FlipSign}(|\psi\rangle))
\]

Finally, applying the Hadamard on the control qubit to complete the phase kickback process:

\[
H \left(\dfrac{1}{\sqrt{2}}(|0\rangle \otimes |\psi\rangle + |1\rangle \otimes
\text{FlipSign}(|\psi\rangle))\right)
\]

When we sample, we effectively measure the final state of the control qubit, which gives us insight
into the key.

\vsp

This equation outlines the process described by the program in PennyLane, depicting how the initial
state is manipulated to exhibit phase kickback.
\end{airesult}


\begin{ai}
Can you break down the last equation, namely:

\[
H \left(\dfrac{1}{\sqrt{2}}(|0\rangle \otimes |\psi\rangle + |1\rangle \otimes
\text{FlipSign}(|\psi\rangle))\right)
\]

I'd like you to apply this Hadamard to the first qubit and write down the result
\end{ai}

\begin{airesult}
Certainly! Let's break down the equation step by step by applying the Hadamard gate to the control
qubit component inside the expression:

\vsp

Our initial expression is:

\[
H \left(\dfrac{1}{\sqrt{2}} \left( |0\rangle \otimes |\psi\rangle + |1\rangle \otimes
\text{FlipSign}(|\psi\rangle) \right) \right)
\]

The Hadamard gate \( H \) applied on a single qubit can be represented as:

\[
H = \dfrac{1}{\sqrt{2}} \begin{pmatrix} 1 & 1 \\ 1 & -1 \end{pmatrix}
\]

Applying this Hadamard gate to our expression means we are operating on the control qubit states
\(|0\rangle\) and \(|1\rangle\):

\[
H |0\rangle = \dfrac{1}{\sqrt{2}} (|0\rangle + |1\rangle)
\]
\[
H |1\rangle = \dfrac{1}{\sqrt{2}} (|0\rangle - |1\rangle)
\]

Substituting these transformations into our original expression, we get:

\[
= \dfrac{1}{\sqrt{2}} \left( H|0\rangle \otimes |\psi\rangle + H|1\rangle \otimes
\text{FlipSign}(|\psi\rangle) \right)
\]

Applying the transformations:

\[
= \dfrac{1}{\sqrt{2}} \left( \dfrac{1}{\sqrt{2}} (|0\rangle + |1\rangle) \otimes |\psi\rangle
+ \dfrac{1}{\sqrt{2}} (|0\rangle - |1\rangle) \otimes \text{FlipSign}(|\psi\rangle) \right)
\]

Simplifying further:

\[
= \dfrac{1}{2} \left( |0\rangle \otimes |\psi\rangle + |1\rangle \otimes |\psi\rangle
+ |0\rangle \otimes \text{FlipSign}(|\psi\rangle) - |1\rangle \otimes \text{FlipSign}(|\psi\rangle)
\right)
\]

Finally, collecting like terms:

\[
= \dfrac{1}{2} \left( |0\rangle \otimes (|\psi\rangle + \text{FlipSign}(|\psi\rangle))
+ |1\rangle \otimes (|\psi\rangle - \text{FlipSign}(|\psi\rangle)) \right)
\]

This is the result after applying the Hadamard to the first qubit in your equation. The outcome will
depend on the nature of the \textit{FlipSign} operation applied to \(|\psi\rangle\).
\end{airesult}

\subsection{The case of matching key}

\begin{ai}
Great! Now consider that $FlipSign$ changes the sign of the state to the opposite: $|\psi\rangle
\mapsto -|\psi\rangle$.
\end{ai}

\begin{airesult}
If the \textit{FlipSign} operation changes the sign of the state, i.e.,

\[
\text{FlipSign}(|\psi\rangle) \mapsto -|\psi\rangle
\]

we can substitute this into our expression that we derived:

\vsp

The modified expression becomes:

\[
= \dfrac{1}{2} \left( |0\rangle \otimes (|\psi\rangle + (-|\psi\rangle))
+ |1\rangle \otimes (|\psi\rangle - (-|\psi\rangle)) \right)
\]

Simplifying each term:

- The first term simplifies to:

\[
|0\rangle \otimes (|\psi\rangle - |\psi\rangle) = |0\rangle \otimes 0
\]

- The second term simplifies to:

\[
|1\rangle \otimes (|\psi\rangle + |\psi\rangle) = |1\rangle \otimes 2|\psi\rangle
\]

The entire expression becomes:

\[
= \dfrac{1}{2} \left( |0\rangle \otimes 0 + |1\rangle \otimes 2|\psi\rangle \right)
\]

\[
= |1\rangle \otimes |\psi\rangle
\]

\vsp

Thus, after applying the Hadamard to the first qubit and considering the \textit{FlipSign}
operation, the final state is \(|1\rangle \otimes |\psi\rangle\). This indicates that the effect of
the \textit{FlipSign} operation is crucial in determining the final measured state.
\end{airesult}



\subsection{The case of mismatching key}

\begin{ai}
Great! Now lets take a step back and consider that $FlipSign$ just returns the original state.
\end{ai}

\begin{airesult}
If the \textit{FlipSign} operation does not alter the state, meaning:

\[
\text{FlipSign}(|\psi\rangle) \mapsto |\psi\rangle
\]

we can substitute this into our previously derived expression:

\vsp

The original expression becomes:

\[
= \dfrac{1}{2} \left( |0\rangle \otimes (|\psi\rangle + |\psi\rangle)
+ |1\rangle \otimes (|\psi\rangle - |\psi\rangle) \right)
\]

Simplifying each term:

- The first term simplifies to:

\[
|0\rangle \otimes (|\psi\rangle + |\psi\rangle) = |0\rangle \otimes 2|\psi\rangle
\]

- The second term simplifies to:

\[
|1\rangle \otimes (|\psi\rangle - |\psi\rangle) = |1\rangle \otimes 0
\]

The entire expression becomes:

\[
= \dfrac{1}{2} \left( |0\rangle \otimes 2|\psi\rangle + |1\rangle \otimes 0 \right)
\]

\[
= |0\rangle \otimes |\psi\rangle
\]

\vsp

Thus, when \textit{FlipSign} returns the original state, the final state after applying the Hadamard
to the first qubit is \(|0\rangle \otimes |\psi\rangle\). This demonstrates how the absence of sign
change affects the resulting quantum state.
\end{airesult}

