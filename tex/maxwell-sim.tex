
\section{Chapter 1}

Base formulas:

\[
\begin{aligned}
  \frac{\partial E}{\partial t} &= \frac{1}{\epsilon_0} \nabla \times H, \\
  \frac{\partial H}{\partial t} &= -\frac{1}{\mu_0} \nabla \times E.
\end{aligned}
\]


Note: $\nabla$ in Cartesian coordinates:
\[
\nabla \times \mathbf{F} = \left( \frac{\partial F_z}{\partial y} - \frac{\partial F_y}{\partial
z} \right) \mathbf{i}
+ \left( \frac{\partial F_x}{\partial z} - \frac{\partial F_z}{\partial x} \right) \mathbf{j}
+ \left( \frac{\partial F_y}{\partial x} - \frac{\partial F_x}{\partial y} \right) \mathbf{k}
\]
where \(\mathbf{F}\) is a vector field, and \(\mathbf{i}, \mathbf{j}, \mathbf{k}\) are unit
vectors in the \(x\), \(y\), and \(z\) directions, respectively.

\vsp

Below are the equations of a plane wave traveling in the $z$ direction with the
electric field oriented in the $x$ direction and the magnetic field oriented in the
$y$ direction

\[
\begin{aligned}
\frac{\partial E_x}{\partial t} &= -\frac{1}{\epsilon_0} \frac{\partial H_y}{\partial z}, \\
\frac{\partial H_y}{\partial t} &= -\frac{1}{\mu_0} \frac{\partial E_x}{\partial z}.
\end{aligned}
\]

After the central difference approximation:

\[
\begin{aligned}
\frac{E_x^{n + 1/2}(k) - E_x^{n - 1/2}(k)}{\Delta t} &= -\frac{1}{\epsilon_0} \frac{H_y^n(k + 1/2) - H_y^n(k - 1/2)}{\Delta x}, \\
\frac{H_y^{n + 1}(k + 1/2) - H_y^n(k + 1/2)}{\Delta t} &= -\frac{1}{\mu_0} \frac{E_x^{n + 1/2}(k + 1) - E_x^{n + 1/2}(k)}{\Delta x}.
\end{aligned}
\]

Note: a comment from \cite{houle2019fdtd}: It might seem more sensible to use $\Delta{z}$ as the
incremental step because in this case we are going in the $z$ direction. However, $\Delta{x}$ is so
commonly used for a spatial increment that we will use $\Delta{x}$.

The formulation of the above equations assume that the $E$ and $H$ fields are interleaved in both
space and time. $H$ uses the arguments $k + 1/2$ and $k - 1/2$ to indicate that the $H$ field values
are assumed to be located between the $E$ field values. Similarly, the $n + 1/2$ or $n - 1/2$
superscript indicates that it occurs slightly after or before $n$, respectively.

\begin{python}
import numpy as np
from math import exp
from matplotlib import pyplot as plt
\end{python}

\begin{python}
ke = 200
ex = np.zeros(ke)
hy = np.zeros(ke)
# Pulse parameters
kc = int(ke / 2)
t0 = 40
spread = 12

nsteps = 100
# Main FDTD Loop
for time_step in range(1, nsteps + 1):
  # Calculate the Ex field
  for k in range(1, ke):
    ex[k] = ex[k] + 0.5 * (hy[k - 1] - hy[k])
  # Put a Gaussian pulse in the middle
  pulse = exp(-0.5 * ((t0 - time_step) / spread) ** 2)
  ex[kc] = pulse
  # Calculate the Hy field
  for k in range(ke - 1):
    hy[k] = hy[k] + 0.5 * (ex[k] - ex[k + 1])

# Plot the outputs as shown in Fig. 1.2
plt.rcParams['font.size'] = 12
plt.figure(figsize=(8, 3.5))
plt.subplot(211)
plt.plot(ex, color='k', linewidth=1)
plt.ylabel('E$_x$', fontsize='14')
plt.xticks(np.arange(0, 201, step=20))
plt.xlim(0, 200)
plt.yticks(np.arange(-1, 1.2, step=1))
plt.ylim(-1.2, 1.2)
plt.text(100, 0.5, 'T = {}'.format(time_step),
horizontalalignment='center')
plt.subplot(212)
plt.plot(hy, color='k', linewidth=1)
plt.ylabel('H$_y$', fontsize='14')
plt.xlabel('FDTD cells')
plt.xticks(np.arange(0, 201, step=20))
plt.xlim(0, 200)
plt.yticks(np.arange(-1, 1.2, step=1))
plt.ylim(-1.2, 1.2)
plt.subplots_adjust(bottom=0.2, hspace=0.45)
plt.savefig('img/maxwell-sim-1.png')
\end{python}

\begin{comment}
\begin{result}
\end{result}
\end{comment}

\begin{center}
\includegraphics[width=0.8\linewidth]{img/maxwell-sim-1.png}
\end{center}



\section{Discussion}


\begin{ai}

Hi model. Please, describe the below Maxwell's equations. What is $H$, what is $E$. Please define
the $\nabla$ operator.

\[
\begin{aligned}
  \frac{\partial E}{\partial t} &= \frac{1}{\epsilon_0} \nabla \times H, \\
  \frac{\partial H}{\partial t} &= -\frac{1}{\mu_0} \nabla \times E.
\end{aligned}
\]

\end{ai}

\begin{airesult}
Hi! Let's explore the Maxwell's equations you've provided, which describe the fundamental behavior
of electric and magnetic fields.

\vsp

The equations are:

\[
\begin{aligned}
  \frac{\partial E}{\partial t} &= \frac{1}{\epsilon_0} \nabla \times H, \\
  \frac{\partial H}{\partial t} &= -\frac{1}{\mu_0} \nabla \times E.
\end{aligned}
\]

\vsp

Here's a breakdown of the components involved:

\begin{itemize}
\item \textbf{E (Electric Field):}

  \begin{itemize}
  \item Denoted by \( E \), it represents the electric field vector, a fundamental field that
describes the force experienced by a charged particle in space.
  \item It is measured in volts per meter (V/m) and describes the magnitude and direction of the
electric force per unit charge.
  \end{itemize}

\item \textbf{H (Magnetic Field Intensity):}

  \begin{itemize}
  \item Denoted by \( H \), it represents the magnetic field intensity vector. It's related to the
magnetic field \( B \) through the material's magnetic properties.
  \item It is measured in amperes per meter (A/m) and is often used in disciplines involving
magnetized materials.
  \end{itemize}

\item \textbf{The \(\nabla\) Operator (Nabla or Del):}

  \begin{itemize}
  \item The \(\nabla\) symbol is a vector differential operator known as the del or nabla operator.
  \item When applied to a vector field (like \( H \) or \( E \)) in the form \(\nabla \times\), it
represents the curl of the field.
  \item The curl of a vector field gives a measure of the field's tendency to circulate around a
point, effectively capturing the rotation or swirling of the field.

  \item For example, in Cartesian coordinates:
  \[
  \nabla \times \mathbf{F} = \left( \frac{\partial F_z}{\partial y} - \frac{\partial F_y}{\partial
z} \right) \mathbf{i}
  + \left( \frac{\partial F_x}{\partial z} - \frac{\partial F_z}{\partial x} \right) \mathbf{j}
  + \left( \frac{\partial F_y}{\partial x} - \frac{\partial F_x}{\partial y} \right) \mathbf{k}
  \]
  where \(\mathbf{F}\) is a vector field, and \(\mathbf{i}, \mathbf{j}, \mathbf{k}\) are unit
vectors in the \(x\), \(y\), and \(z\) directions, respectively.
  \end{itemize}

\item \textbf{Constants:}

  \begin{itemize}
  \item \(\epsilon_0\) is the permittivity of free space, a measure of how an electric field affects
and is affected by a medium. Its approximate value is \(8.854 \times 10^{-12} \ \text{F/m}\) (farads
per meter).
  \item \(\mu_0\) is the permeability of free space, indicating the ability of a medium to support
the formation of a magnetic field. Its value is approximately \(4\pi \times 10^{-7} \ \text{H/m}\)
(henrys per meter).
  \end{itemize}

\end{itemize}

\vsp

These equations describe the dynamics of electromagnetic fields in space, with the curl operations
indicating how changes in one type of field (electric or magnetic) can generate or be related to
changes in the other field over time. This intertwined behavior of \(E\) and \(H\) is critical in
explaining electromagnetic wave propagation, transmission lines, and antennas, among other
phenomena.
\end{airesult}
